% Combing MPEG-TS and BitTorrent
% - Problems and suggested solutions
% 
% Author: Erik Helin
% Date: April 2010
% License: Creative Commons Attribution license, see
% http://creativecommons.org/licenses/by/3.0/us

\documentclass[11pt,oneside,a4paper,onecolumn]{article}
\usepackage{url}
\begin{document}

% BEGIN TITLE
\title{Combining MPEG-TS and BitTorrent\\
       \large{Problems and Suggested Solutions}}
\author{Erik Helin\\\url{ehelin@kth.se}}
\maketitle
% END TITLE

\section{Background}
The standard protocol today for digital 
video broadcasting (DVB) is called 
MPEG Transport Stream (MPEG-TS).
DVB systems distribute the data by different approaches, 
for example satellite, cable and terrestial television\cite{dvb}.
Today, the internet protocol (IP) is emerging as another medium for distributing television. 
A common technique for distributing TV in access networks is multicast.
The use of multicast eases the load on the servers providing the media
since the data is replicated in the routers\cite{multicast}.
One of the most widely used protocols for transporting the television
data via multicast is MPEG-TS.
The drawback of only using multicast is that services such as
video-on-demand are hard to implement. A possible solution to this
problem is to combine MPEG-TS with another transportation technique.

This article will examine the possible use of BitTorrent together with 
multicast. BitTorrent is a protocol which was created by Bram Cohen in
2001\cite{bittorrent}. BitTorrent makes it possible for clients in a
network to efficiently share data with each other. This is because
the BitTorrent protocol splits the file into smaller pieces. For each
piece, a hash is created. Therefore, a client can share data with the
rest of the network, even if he/she only has a small part of the original
file. Due to this, large files can be effectively distributed via
BitTorrent. Another advantage of using BitTorrent to share files is
that a server having the data isn't required. Since the clients share
the data with each other, the server only needs to distribute the ip
addresses of the clients to the clients. This is done by a so-called
tracker.

However, when combining MPEG-TS with BitTorrent, several problems
arise.

\section{Problems}
The idea when combining MPEG-TS and BitTorrent is that a video will be
broadcast in access network using multicast and MPEG-TS. Once a client
has received data from the multicast stream, it should be able to share this
data with other clients using BitTorrent. Therefore, any new clients
entering the network will be able to watch the broadcast from the
beginning, by downloading data from other clients via BitTorrent. This
will create a "catch-up" effect, without requiring any additional
servers providing the data. 

\subsection{Problem 1}
\label{sec:p1}
The first major problem when combining BitTorrent and MPEG-TS is that
the 
BitTorrent protocol needs to have the entire file beforehand, otherwise
it will not be able to create the hashes for all the pieces. This is
because the hash depends on the content of the piece. Since TV data could originate from a live source, such as a soccer game,
it's impossible to have the data in advance. However, this
article will be focusing on the use of BitTorrent in an access
network where the distributed data will be known prior to broadcasting, so
 a solution to this problem will not be mentioned.

\subsection{Problem 2}
The second major problem when combining MPEG-TS and BitTorrent is how to
find out which BitTorrent piece a received MPEG-TS packet belongs
to. If the data is meant
to be shared via BitTorrent, the received MPEG-TS packet must be added
to a piece. Once all MPEG-TS packets belonging to a piece have been received,
the piece will be hashed and then shared.
The MPEG-TS protocol was meant to be used for live distribution
of video. Therefore, the protocol does not contain any kind of "state".
It is impossible to know, when receiving an MPEG-TS packet, from which
part of the original file the packet comes from. Therefore, it's also
impossible, from an MPEG-TS packet alone, to know which piece the packet
originated.

\subsection{Problem 3}
The third major problem is the integrity of the data transported via
MPEG-TS. For a client to be able to share data via BitTorrent, the
recieved data must be an exact bit-for-bit copy of the original file. If
this is not the case, the hash for the downloaded piece will not be the
same as the one created before broadcasting. BitTorrent does not allow
sharing of data with different hashes, so therefore the client will not be
able to share the received data. A packed transported by MPEG-TS is exactly 188 bytes. At least once
each 100 ms, this packages contains a program clock reference (PCR).
The PCR consists of 42 bits. The first 33 bits are based on a 90kHz
hardware clock, and the last 9 bits are based on a 27MHz clock.
Usually when a video is sent using MPEG-TS, the values of the PCR will
be renewed. Therefore, a movie sent with the help of MPEG-TS will not be
the same if it is broadcasted twice, since the second time, there will
be new PCRs. The two different movies will look identical with regards
to playback, but they will differ on a bit-level. 

\section{Suggested solutions}
\subsection{Suggested solution 1}
As mentioned in section \ref{sec:p1}, a solution for the first problem will not be discussed in this article.
For more information about a solution to this problem, see \url{http://www.tribler.org/trac}.

\subsection{Suggested solution 2}
I will here describe two possible ways to solve problem number two, 
and discuss their advantages and disadvantages.

The first solution uses the fact that the file is known before it is being distributed. 
This solution also assumes that it is possible to make modifications to the file before 
distributing it. Because of these two assumptions, one can insert data into the file 
telling which piece a part of the file belongs to. Since there is no lower limit on 
the payload data of an MPEG-TS packet, it is hard to know how often the added piece data would 
have to be inserted into the file. However, this is not strictly necessary. The downloaded data 
could be searched for the appearance of piece information, and then one would be able to deduce
which piece the data belongs to. The drawbacks of this solution is that it requires modification of the file. 
This means that a normal decoder will not be able to decode the file and retrieve the "raw" 
movie data. Therefore, a custom decoder will have to be written, something which is hard to do 
and error prone. An advantage of this solution is that only one file has to be transmitted.

The second solutions is based on the assumption that it is possible to send additional data to 
the client via other sources, such as HTTP. This solution is also dependant on a solution to 
problem number three. This solution also depends on the fact that the shared data will be a 
saved copy of the MPEG-TS stream, not the movie transported by the MPEG-TS stream.
The idea is to save a copy of the MPEG-TS stream by sending and then letting a receiver only 
save the packages to disc. This could be done at high speeds as long as there is no risk of packet loss.
The saved copy of the stream will then be retransmitted both with MPEG-TS, and also shared via 
BitTorrent. The saved copy of the stream will be hashed and prepared for distribution via BitTorrent.
The BitTorrent pieces will be created in a way so that the size of each piece is a multiple of 188 bytes.
The saved copy of the stream will also be split into 188 bytes chunks (the same size as the packets distributed the second time).
Since each BitTorrent piece is a multiple of 188 bytes, 
each BitTorrent piece will contain a fixed amount of these 188 bytes chunks. 

\subsection{Suggested solution 3}
I have no complete suggestion for a solution to problem number three. One possible way to achieve integrity of the transported file is to 
zero out the PCR field of the MPEG-TS packet. This modified MPEG-TS will now be bit-for-bit equal to another MPEG-TS carrying the exact same
payload. Therefore, an MPEG-TS file with zeroed PCRs could be shared via BitTorrent. However, a decoder would likely fail at decoding the MPEG-TS file
with zeroed PCR, since the PCR is often used in the decoding process. A solution to this would be to store the original value of the PCR and reapply 
for the decoding process.

The advantages of this solution is that it would be possible to transfer MPEG-TS streamed packets via BitTorrent. The disadvantages are however
significant, since a MPEG-TS decoder would have to be written (or bought) to be able to change the PCR. It is also hard to apply the PCR once again
for the decoding process. If one could find a multicast server which did not change the PCR when broadcasting the file, the problem would disappear. 
There are however multicast servers who do change the PCR, for example VideoLAN, so if one wants full compatibility with all multicast
servers, this problem must be solved.

\section{Conclusion}
Unfortunately, there are no easy solutions to the problems presented in this article. The reasons for this are, as mentioned earlier, largely due to
the fact that an MPEG-TS has no information regarding it's own "state". Of course, there might be other solutions to the problem than the ones suggested
here. An alternative approach is to write a protocol similar to BitTorrent, which is easier to combine with MPEG-TS. I do not know of solutions
other than the ones presented here which combine standard BitTorrent and MPEG-TS.


\section{Conclusion}

\begin{thebibliography}{9}
\bibitem{dvb}
    Wikipedia,
    \emph{Digital Video Broadcasting},
    \url{http://en.wikipedia.org/wiki/Digital_Video_Broadcasting},
    May, 2010
\bibitem{multicast}
    Wikipedia,
    \emph{Multicast},
    \url{http://en.wikipedia.org/wiki/Multicast},
    May, 2010
\bibitem{bittorrent}
    Wikipedia,
    \emph{BitTorrent (protocol)},
    \url{http://en.wikipedia.org/wiki/BitTorrent_(protocol)},
    May, 2010
\end{thebibliography}
\end{document}


